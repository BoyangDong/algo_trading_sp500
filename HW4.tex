\documentclass[12pt]{article}
\usepackage[left=0.5in,top=0.5in,right=0.5in,bottom=0.5in,nohead]{geometry}
\title{HW4 - FINM 35910}
\author{\vspace{-10ex}}
\date{\vspace{-8ex}}
\begin{document}
\maketitle
\setlength{\parindent}{0pt}
We use the moving average and the momentum indicator. Thus, the key parameters are then
the period of the moving average and the momentum indicator, also how many ticks they should
keep their trend consistent. We use grid search to find the optimal parameters, based on both
the parameter values and the final net return(P\&L). Some parameters have really high P\&L but 
we think that it is not stable enough and will not perform well across other data, so it's a tradeoff
between the training data and the testing data, in other words avoiding overfitting. 
\medbreak
We use the most recent data from May.2020 to now to optmize the parameters, and the best parameters
we get are 4 and 4. First 4 indicates the period we use to calculate the moving average as well
as the gap we use to calculate the momentum, and the second 4 indicates the tick number that the trends
have to keep consistent. The training period gives a net P\&L that's 139 times our starting fund,
reaching 13900\% of return in one year. 
\medbreak
Besides this, we also try using the machine learning techniques to predict with these two
indicators whether at any given hour, the relation between a closing price and an opening price. 
If the closing is higher, we should then long at the start of the hour, and vice versa. 
This approach doesn't yield a good result even after caliberation, with accuracy only at around 50\%, 
and running the trading algorithm yields negative return in our period. Possible reasons include:
we might need more indicators to strengthen the model; the after-day trading period may affect
the correlation between the previous closing data and the next day opening data, and thus indicators 
calculated using previous data might lose accuracy. In general, our machine learning models are not well 
adapted to the market yet, whether it's random forest, linear support vector machine, or neural network. 

\end{document}